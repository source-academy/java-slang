\section{Syntax}

A Java program is a \emph{compilation-unit}, defined using Backus-Naur Form\footnote{
We adopt Henry Ledgard's BNF variant that he described in
\emph{A human engineered variant of BNF}, ACM SIGPLAN Notices, Volume 15 Issue 10,
October 1980, Pages 57-62. In our grammars, we use \textbf{\texttt{bold}} font for keywords,
{\it italics} for syntactic variables, $\epsilon$ for nothing,
$x\ |\ y$ for $x$ or $y$, $[\ x\ ]$ for an optional $x$,
$ x ...$ for zero or more repetitions of $x$, and $(\ x\ )$ for clarifying the structure of BNF expressions.}
as follows:
