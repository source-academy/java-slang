% !TEX root = ./jvm.tex
\input source_header.tex
\usepackage{titlesec}

\setcounter{secnumdepth}{4}

\newcommand{\Rule}[2]{\genfrac{}{}{0.7pt}{}{{\setlength{\fboxrule}{0pt}\setlength{\fboxsep}{3mm}\fbox{$#1$}}}{{\setlength{\fboxrule}{0pt}\setlength{\fboxsep}{3mm}\fbox{$#2$}}}}

\newcommand{\TruE}{\textbf{\texttt{true}}}
\newcommand{\FalsE}{\textbf{\texttt{false}}}
\newcommand{\AndOp}{\texttt{\&\&}}
\newcommand{\OrOp}{\texttt{||}}
\newcommand{\ThenOp}{\texttt{?}}
\newcommand{\ElseOp}{\texttt{:}}
\newcommand{\Rc}{\texttt{\}}}
\newcommand{\Lc}{\texttt{\{}}
\newcommand{\Rp}{\texttt{)}}
\newcommand{\Lp}{\texttt{(}}
\newcommand{\Fun}{\textbf{\texttt{function}}}
\newcommand{\Let}{\textbf{\texttt{let}}}
\newcommand{\Return}{\textbf{\texttt{return}}}
\newcommand{\Const}{\textbf{\texttt{const}}}
\newcommand{\If}{\textbf{\texttt{if}}}
\newcommand{\Else}{\textbf{\texttt{else}}}
\newcommand{\Bool}{\texttt{boolean}}
\newcommand{\Number}{\texttt{number}}
\newcommand{\String}{\texttt{string}}
\newcommand{\Undefined}{\texttt{undefined}}
\newcommand{\Null}{\texttt{null}}
\newcommand{\Any}{\texttt{any}}
\newcommand{\Void}{\texttt{void}}
\newcommand{\Pred}{\textit{Pred}}
\newcommand{\type}{\textit{type}}
\newcommand{\polytype}{\textit{polytype}}
\newcommand{\predtype}{\textit{predtype}}
\newcommand{\ExtractPos}{\ensuremath{\textit{Extract}^+}}
\newcommand{\ExtractNeg}{\ensuremath{\textit{Extract}^-}}

\begin{document}
	%%%%%%%%%%%%%%%%%%%%%%%%%%%%%%%%%%%%%%%%%%%%%%%
	\docheader{Java}{Source}{}{Martin Henz, Yap Hock Chuan Roland, Chuan Hao Chang}
	%%%%%%%%%%%%%%%%%%%%%%%%%%%%%%%%%%%%%%%%%%%%%%%

\input source_intro.tex

This JVM follows the Java SE8 Edition \href{https://docs.oracle.com/javase/specs/jvms/se8/html/index.html}{JVM Specification}.

This documentation contains limitations not covered in the official specifcations above, and is sectioned
in a way that mirrors the official specification document. For general documentation of the JVM, refer to the official specification.

\stepcounter{section}

\section{Structure of the JVM}

\addtocounter{subsection}{7}
\subsection{Floating-Point Arithmetic}
\stepcounter{subsubsection}
\subsubsection{Floating-Point Modes}

The \texttt{ACC\_STRICT} flag is not implemented.

\addtocounter{subsection}{3}

\subsection{Class Libraries}

The JVM is expected to support implementation of class libraries.

Since the standard library is largely left unimplemented, the full scope of what the JVM needs to support is not known.

\stepcounter{section}
\section{The class file format}

\subsection{ClassFile Structure}

The following values are not verified:

\begin{enumerate}
\item \texttt{magic}
\item \texttt{minor\_version}
\item \texttt{major\_version}
\item \texttt{constant\_pool\_count}
\item \texttt{interfaces\_count}
\item \texttt{fields\_count}
\item \texttt{methods\_count}
\item \texttt{attributes\_count}
\end{enumerate}

\addtocounter{subsection}{2}
\subsection{Constant Pool}

The below constants may not be fully implemented.

\begin{enumerate}
\item \texttt{CONSTANT\_MethodHandle\_info}: Relies on \texttt{MethodHandleNatives::linkMethodHandleConstant} for resolution, some native methods invoked by \texttt{linkMethodHandleConstant} may not be implemented.
\item \texttt{CONSTANT\_MethodType\_info}: Relies on \texttt{MethodHandleNatives::findMethodHandleType} for resolution, some native methods invoked by \texttt{findMethodHandleType} may not be implemented.
\item \texttt{CONSTANT\_InvokeDynamic\_info}: Relies on \texttt{MethodHandleNatives::linkCallSite} for resolution, some native methods invoked by \texttt{linkCallSite} may not be implemented.
\end{enumerate}

There is currently a bug when resolving \texttt{InvokeDynamic} Constants, where the resolved callsite invokes the original private method instead of the JVM provided bridge method.

An additional \texttt{NestHost} Attribute is added to the lambda class so it passes the access control check.

\addtocounter{subsection}{2}
\subsection{Attributes}

The below attributes are not supported, and will be present in the class data as an \texttt{UnhandledAttribute}.

\begin{enumerate}
	\item \texttt{RuntimeVisibleAnnotations}
	\item \texttt{RuntimeInvisibleAnnotations}
	\item \texttt{RuntimeVisibleParameterAnnotations}
	\item \texttt{RuntimeInvisibleParameterAnnotations}
	\item \texttt{RuntimeVisibleTypeAnnotations}
	\item \texttt{RuntimeInvisibleTypeAnnotations}
	\item \texttt{AnnotationDefault}
	\item \texttt{MethodParameters}
\end{enumerate}

\addtocounter{subsection}{2}
\subsection{Verification of class Files}

Verification is skipped entirely.

\section{Loading, Linking, and Initializing}

\stepcounter{subsection}
\subsection{Java Virtual Machine Startup}

The JVM startup process is as follows:

\begin{enumerate}
	\item Essential classes are loaded (\texttt{Object},\texttt{Thread},\texttt{System},\texttt{Class},\texttt{ClassLoader},\texttt{ThreadGroup},\texttt{Unsafe})
	\item \texttt{ThreadGroup} and \texttt{Thread} classes are initialized, then the initial thread is instantiated.
	\item \texttt{System} is initialized by calling \texttt{System::initializeSystemClass}.
	\item The system class loader is initialized by calling \texttt{ClassLoader::getSystemClassLoader}.
	\item The main class is loaded with the system class loader. the \texttt{main} method is run.
\end{enumerate}

Initialization will fail if the above classes and its dependencies cannot be found.

\subsection{Creation and Loading}
\addtocounter{subsubsection}{3}
\subsubsection{Loading Constraints}

The JVM does not ensure loading constraints are checked. The same class, loaded by 2 different class loaders, are treated as different classes.

\subsection{Linking}

Symbolic references are implemented as lazy resolution.

\subsubsection{Verification}

This is skipped.

\addtocounter{subsubsection}{1}
\subsubsection{Resolution}

\addtocounter{paragraph}{4}
\paragraph{Method Type and Method Handle Resolution}

Method Type and Method Handles are resolved using Java methods. These methods may rely on native methods whose implementations are not fully implemented/tested.
\begin{itemize}
	\item Method types are resolved by calling \texttt{MethodHandleNatives::findMethodHandleType}.
	\item Method handles are resolved by calling \texttt{MethodHandleNatives::linkMethodHandleConstant}.
\end{itemize}

\paragraph{Call Site Specifier Resolution}

Call Site Specifiers are resolved by calling \texttt{MethodHandleNatives::linkCallSite}.

\texttt{linkCallSite} may invoke native methods whose implementations are not fully implemented/tested.

\subsubsection{Access Control}

Access control also implements the nest mate test from the \href{https://docs.oracle.com/javase/specs/jvms/se11/html/jvms-5.html#jvms-5.4.4}{SE11 edition JVM specifications}. 
There is a bug where the anonymous inner classes from lambda creation are invoking private methods of the original class directly instead of the synthetic bridge methods we generate.
To work around this, we add a \texttt{NestHost} attribute to the created class, and implement the nest mate access control logic as a hacky solution.

\section{The Java Virtual Machine Instruction Set}

\begin{itemize}
\item \texttt{BREAKPOINT}: Not implemented. Throws an error.
\item \texttt{IMPDEP1}: Not implemented. Throws an error.
\item \texttt{IMPDEP2}: Not implemented. Throws an error.
\end{itemize}


\end{document}